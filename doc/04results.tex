\section{Tournaments}
To verify how well each bot performs, we have the different agents compete against each other in a tournament. To get a reference on whether a bot is reasonably configured, we also let a random bot compete in each case, which makes arbitrary moves. It is clear that all bots should clearly win against this agent. To get a meaningful result, we let the bots face each other 1000 times in each encounter.

\subsection{Participating bots}
Monte, Minimax (Reader+Minimax ?), Random

\subsection{Best guess configuration}
In a first pass, we tried to obtain a solid configurations of the bots through empirical observation. For Minimax we found a maximum depth of 7 (d7) and for Monte Carlo a maximum number of simulations of 1'000 (s1000) as a reasonable basic configuration.

We let the bots with these basic configuration compete against each other:

\begin{table}[ht]
  \renewcommand{\arraystretch}{2}
  \arrayrulecolor{white}
  \begin{center}
    \rowcolors{1}{\seccolor!10}{\seccolor!10}
    \begin{threeparttable}
      \begin{tabular}{c|c}
        \rowcolor{\seccolor!50}
        Matchup & Result \\\bfhmidline
        Random vs Minimax (d7) & 0 : 1000 \\\bfhmidline
        Random vs Monte (s1000) & 304 : 696 (7:23) \\\bfhmidline
        Minimax (d7) vs Monte (s1000) & 1000 : 0 \\\bfhmidline
      \end{tabular}
      \caption{Basic configuration}
    \end{threeparttable}
    \label{tab:table1}
  \end{center}
\end{table}

\subsection{Minimax with more depth}
We increase the max depth for the Minimax algorithm to 9 and verify if this improves the performance against the other bots.

\begin{table}[ht]
  \renewcommand{\arraystretch}{2}
  \arrayrulecolor{white}
  \begin{center}
    \rowcolors{1}{\seccolor!10}{\seccolor!10}
    \begin{threeparttable}
      \begin{tabular}{c|c}
        \rowcolor{\seccolor!50}
        Matchup & Result \\\bfhmidline
        Random vs Minimax (d9) & 0 : 0 \\\bfhmidline
        Random vs Monte (s1000) & 0 : 0 \\\bfhmidline
        Minimax (d9) vs Monte (s1000) & 0 : 0 \\\bfhmidline
      \end{tabular}
      \caption{Increased max depth for Minimax}
    \end{threeparttable}
    \label{tab:table1}
  \end{center}
\end{table}

\subsection{More simulations for Monte Carlo}
We increase the max depth for the Minimax algorithm to 9 and verify if this improves the performance against the other bots.

\begin{table}[ht]
  \renewcommand{\arraystretch}{2}
  \arrayrulecolor{white}
  \begin{center}
    \rowcolors{1}{\seccolor!10}{\seccolor!10}
    \begin{threeparttable}
      \begin{tabular}{c|c}
        \rowcolor{\seccolor!50}
        Matchup & Result \\\bfhmidline
        Random vs Minimax (d9) & 0 : 0 \\\bfhmidline
        Random vs Monte (s2000) & 0 : 0 \\\bfhmidline
        Minimax (d9) vs Monte (s2000) & 0 : 0 \\\bfhmidline
      \end{tabular}
      \caption{Increased max simulations for Minimax}
    \end{threeparttable}
    \label{tab:table1}
  \end{center}
\end{table}


\begin{table}[ht]
  \renewcommand{\arraystretch}{2}
  \arrayrulecolor{white}
  \begin{center}
    \rowcolors{1}{\seccolor!10}{\seccolor!10}
    \begin{threeparttable}
      \begin{tabular}{lll}
        \rowcolor{\seccolor!50}
        1 & 2 & 3 \\\bfhmidline
        1 & 2 & 3 \\\bfhmidline
        1 & 2 & 3 \\\bfhmidline
        1 & 2 & 3 \\
      \end{tabular}
      \caption{Table caption}
    \end{threeparttable}
    \label{tab:table1}
  \end{center}
\end{table}
