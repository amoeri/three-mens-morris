\section{Technology}
We took the opportunity to explore some of the newest technologies
and thus the game engine and agents were implemented using an early develepor preview of Python 3.10
\footnote{\url{https://www.python.org/downloads/release/python-3100a7/}}.

\section{Game}

\subsection{Engine}

\begin{lstlisting}[language=Python]
\end{lstlisting}

\subsection{Drivers}
\subsubsection{Play}
play.py is the main script and helps set up games between players.
It can be run with

\lstinline{py play.py PLAYER0 PLAYER1 [NUMBEROFGAMES]}

`PLAYER0`/`PLAYER1` must be identifiers for agents that `play.py` knows of, such as:
- `human` to play through the CLI
- `random` for Rando the random Bot
- `minimax` for minimax algorithm, max depth can be set in `play.py` as the second parameter of the `Minimax()` constructor. It has a default value of `7`.
- `reader+minimax` for a minimax player that uses the opening book `openings.book` and falls back to standard minimax when it reaches an unknown position.
- `reader+random` same principle as `reader+minimax` but falls back to the random bot.
- `monte` for the Monte Carlo search algorithm

`NUMBEROFGAMES` defines how many games will be played. Players will alternate their starting turn. Default value is `1`.
`play.py` will also print whenever a game is started and ended to the terminal, aswell as a score total if more than one game is played.

\subsubsection{Tournament}
`tournament.py` let all bots play against one another and saves the result
in `tournament.results`.
The participating bots are declared in the constant
PARTICIPATING_AGENT_IDENTIFIERS.

`py tournament.py [NUMBEROFGAMES]`

`NUMBEROFGAMES` defines how many games will be played per bot encounter.


\section{Agents}

A series of agents were implemented:
\begin{enumerate}
    \item Human (Command line interface)
    \item Random
    \item Minimax (with Alpha-Beta pruning)
    \item Monte Carlo Tree Search
    \item Reader (Opening book)
\end{enumerate}

Those are further described below.

To work with the game engine an agent must implement:
\begin{enumerate}

    \item A constructor that takes one parameter (id) which will be 0 or 1
    \item Fields `id` (passed to the constructor) and `name` (A string identifying what type of agent it is, ideally also include the id in the name)
    \item A method `getNextMove(self, player, board)` where:
          \begin{itemize}
              \item `player` is the identifier used on the board during this game for that player (either -1 or 1)
              \item `board` is an Object of type Board as defined in `game.py`
              \item the method returns a valid move, which can either be:
                    \begin{enumerate}
                        \item  A tupel `(player, (x, y))` to set a stone (in the starting phase of the game), where `player` is the identifier passed to the method and `x` and `y` are the coordinates of the the position on the board (0, 1, 2), `x` beeing the row and `y` beeing the column
                        \item  A tupel `((x1, y1), (x2, y2))` to move a stone from x1, y1 to x2, y2 during the second phase of the game.
                    \end{enumerate}
              \item It is strongly encouranges to use the boards `legalMoves` method to get a list of the playable moves

          \end{itemize}
\end{enumerate}
There are no checks in place if the moves returned are acutally legal.


\subsection{Human}
The human agent implemented in human.py provids an interface for a human to play
the game and thus an easy way to test the engine as well as other bots.

\subsection{Random}
The random bot was used as a baseline for further exploration and simply always
plays a random legal move.

\subsection{Minimax}

\subsection{Monte Carlo Tree Search}
For the Monte Carlo tree search, we followed the fundamental principle in implementing the heuristic search algorithm. Each round of evaluation of the best possible move consists of four steps:
\begin{itemize}
    \item Selection
    \item Expansion
    \item Simulation
    \item Backpropagation
\end{itemize}
In our implementation, we try to follow the steps mentioned above.

The basic functionality of our algorithm is to play a given number of simulations until a winner is determined. During the simulation phase, a score is kept for every simulation, which defines the move's efficiency for the AI to win. Afterwards, the best possible action, based on the highest score, is then chosen and returned.

\subsection{Reader (Opening Book)}

During testing of our Minimax bot at different depths we realized that at
a depth of 9, the bot consitently played the move (1, (1,1)), indicating that
this was a winning move. Analysis has shown that there is a winning strategy
for the player playing the first stone.

Thus an opening book was written using (implementet in writeopeningbook.py,
written in openings.book) and a wrapper bot reader.py. This bot will wrap around
any of the other bots and play according to the opening book, which results
in always winning when starting as the first player.

As this renders the evaluation of the other bots unuseful, it was disregarded in
the further evaluation.

\section{Logs}
Full game logs, including the board state after each turn will be writting to `games.log`
Tournament results are written to `tournament.results`
